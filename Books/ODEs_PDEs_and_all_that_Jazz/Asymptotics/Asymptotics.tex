\graphicspath{{./Asymptotics/}}


%
%
%
%
\section{Asymptotics and Perturbation Methods}

This section will have a vastly different flavor, in comparison to the previous methods. We are to be no longer concerned with exact solutions to equations, even in the form of an infinite sum, but will set our sights solely on asymptotic or perturbative solutions. You can think of these as \emph{very} good approximation methods, if you've never heard the lingo prior. Well, let's just define these when the time comes, but now be comforted by knowing we will solely search for ``approximate" solutions to equations now. 


%
%
% INTEGRAL ASYMPTOTICS AND LAPLACE'S METHOD
%
%
\subsection{Integral Asymptotics and Laplace's Method}

First we consider the seemingly innocent integral, $I(\lambda)$,
\begin{equation}
\label{laplace_integral} I(\lambda) = \int_a^b f(t) e^{-\lambda g(t) } dt.
\end{equation}

Now what I haven't revealed was properties about $f(t)$, $g(t)$, or $\lambda$. Well, we'll assume $f$ and $g$ are friendly functions, where friendly for us just means that they are smooth enough to be replaced by local Taylor series approximations of appropriately desired degree. 

Well, what about $\lambda$? This where the term \emph{asymptotic} comes into play. We are only concerning ourself here with the behavior of $I(\lambda)$ as $\lambda\rightarrow\infty$. Hence are only on a mission to look at this integral in a limiting behavior scenario. This is the general idea with asymptotic methods, that is, methods that describe limiting behavior of a function, or solution to an equation. We will now briefly give two definitions to solidify this idea.  

\begin{definition}
We say to functions $f_1$ and $f_2$ of a natural number parameter $n$, are \textbf{asymptotically equivalent}, e.g., $f_1(n) \sim f_2(n)$ (as $n\rightarrow\infty$), if and only if $\lim_{n\rightarrow\infty} \frac{f(n)}{g(n)}=1$. This equivalence only holds in the sense of the limit, though.
\end{definition}

For example, consider the functions $g(n) = n^2 - 2n$ and $h(n) = n^2 + 3n + 7$. It clear that that $g(n)\neq h(n) \forall n\in\mathbb{N}$; however, it is clear that $g(n)\sim h(n)$ as $n\rightarrow\infty$, so it is safe to say that $g(n)$ and $h(n)$ are asymptotically equivalent as $n\rightarrow\infty.$ It is paramount to note that we had to state the asymptotic equivalent only holds ``as $n\rightarrow\infty$". In general, the limiting behavior does not have to be as $n\rightarrow\infty$ or even $n\rightarrow0$, it just needs to be at a point of interest for the functions, e.g., where a function may become singular, etc.  Using this idea, we will now explore what an asymptotic expansion of a function is. 

\begin{definition}
An \textbf{asymptotic expansion} of a function $f(x)$ is an expression of $f(x)$ in terms of a series, such that by taking any initial partial sum provides an asymptotic formula for $f(x)$. The successive terms in the series provide an increasing accurate description of $f(x)$, by giving more detail to the growth of $f(x)$. We note that the partial sums of this expansion do not necessarily converge. 
\end{definition} 

Now let's get back to the problem at hand, finding the limiting behavior of (\ref{laplace_integral}) as $\lambda\rightarrow\infty$. To achieve this, we will use \emph{Laplace's Method}.

%
% LAPLACE'S METHOD w/ minimum on interior of integration bounds
%
\subsubsection{Laplace's Method}

Consider (\ref{laplace_integral}), such that $g(t)$ assumes a strict minimum in $[a,b]$ at an interior point, call it $c$. We will also assume that $g'(c)=0$ (duh), $g''(c)>0$ (so the point is a minimum, duh), and $f(c)\neq 0$. Using the secret mathematician trick of adding and subtracting zero, we can rewrite (\ref{laplace_integral}) as

$$I(\lambda) = e^{-\lambda g(c)} \int_a^b f(t) e^{ -\lambda\big[ g(t) - g(c) \big] } dt.$$

This is the part that is crucial for Laplace's Method. Since $g(c)$ is a minimum of $g(t)$ on $[a,b]$, for $\lambda>>1$, the main contribution of $I(\lambda)$ will come from a small neighborhood of $c$. This is more obvious from the definition of $I(\lambda)$ above, rather than where it was first introduced. Hence for $\lambda>>1$, 

\begin{align*}
I(\lambda) &\approx e^{-\lambda g(c)} \int_{c-\epsilon}^{c+\epsilon} f(t) e^{ -\lambda\big[ g(t) - g(c) \big] } dt \\ \\
	& \approx  e^{-\lambda g(c)} f(c) \int_{c-\epsilon}^{c+\epsilon} e^{ -\lambda\big[ g(t) - g(c) \big] } dt  \\ 
\end{align*}

Next we approximate $g(t)$ by its Taylor series centered around $t=c$, 
$$g(t) = g(c) +(t-c)g'(c) + \frac{1}{2!} (t-c)^2 g''(c) + \mathcal{O}( (t-c)^3 ).$$

Substituting the above into the integral equation, we find

\begin{align*}
I(\lambda) &\approx e^{-\lambda g(c)} f(c) \int_{c-\epsilon}^{c+\epsilon} e^{ -\lambda\big[ \big( g(c) +(t-c)g'(c) + \frac{1}{2!} (t-c)^2 g''(c) \big) - g(c) \big] } dt \\ \\
	& \approx  e^{-\lambda g(c)} f(c) \int_{c-\epsilon}^{c+\epsilon} e^{ -\lambda\big[  g'(c)(t-c) + \frac{1}{2!}  g''(c)(t-c)^2 \big] } dt  \\  \\
	&= e^{-\lambda g(c)} f(c) \int_{c-\epsilon}^{c+\epsilon} e^{ - \frac{\lambda}{2}  g''(c)(t-c)^2 } dt  \\  
\end{align*}

since $g'(c)=0$. Recall we said the main contribution of $I(\lambda)$ comes from a neighborhood of $c$. Moreover, on the flip side, we assume the integral's contribution outside that neighborhood, therefore we can expand that neighborhood to all of $\mathbb{R}$, i.e., we now have

\begin{equation*}
I(\lambda)  \approx e^{-\lambda g(c)} f(c) \int_{-\infty}^{\infty} e^{ - \frac{\lambda}{2}  g''(c)(t-c)^2 } dt.
\end{equation*}

Next to integrate the Gaussian-type integral above, we first do the swanky transformation of letting $s = t-c$, and getting

\begin{align*}
I(\lambda) &\approx e^{-\lambda g(c)} f(c) \int_{-\infty}^{\infty} e^{ - \frac{\lambda}{2}  g''(c)s^2 } ds  \\ \\
	&= e^{-\lambda g(c) } f(c) \sqrt{ \frac{2\pi}{\lambda g''(c) } }.
\end{align*}

Thus to leading order we have that 

\begin{equation}
\label{std_laplace} I(\lambda) \sim  e^{-\lambda g(c) } f(c) \sqrt{ \frac{2\pi}{\lambda g''(c) } } \ \ \ \mbox{ as } \lambda\rightarrow\infty.
\end{equation}

This above is the standard Laplace Method's result when the minimum is within the interval $(a,b)$. There are three basic ideas behind this methodology. They are as follows:
\begin{enumerate}
\item Since we are concerned with $\lambda\rightarrow\infty$, the main contribution of $I(\lambda)$ comes from a small neighborhood of the point in which $g(t)$ is at its local minimum, say $t=c$, and can replace the integration bounds from $[a,b] \rightarrow [c-\epsilon, c+\epsilon]$.
\item Within the neighborhood of the minimizer, we can approximate $f(t)$ and $g(t)$ with their Taylor Series until desired order. 
\item Since the main contribution of $I(\lambda)$ is coming from the interval $ [c-\epsilon, c+\epsilon]$, we can extend this interval to $[-\infty,\infty]$, as the regions outside $ [c-\epsilon, c+\epsilon]$ only contribute higher order terms to $I(\lambda)$ as $\lambda\rightarrow\infty$.
\end{enumerate}

We will now go through a very similar calculation to illustrate what happens when the minimum of $g(t)$ in on an end point of the integration bounds, but first we will show the trick for integrating Gaussian-type integrals.


%
% GAUSSIAN INTEGRAL 
%
\subsubsection{Gaussian Integral}

We will show how to explicitly integrate integrals of the form, 
\begin{equation}
\label{gauss_integral} G = \int_{-\infty}^{\infty} e^{-x^2} dx.
\end{equation}

To do this, we will transform this integral to polar coordinates, but first, we will do the magical trick. This is it. Just square (\ref{gauss_integral}), i.e.,

$$G^2 = \int_{-\infty}^{\infty} e^{-x^2} dx  \int_{-\infty}^{\infty} e^{-y^2} dy=  \int_{-\infty}^{\infty}  \int_{-\infty}^{\infty} e^{-(x^2+y^2)} dxdy.$$ 

Now since $G^2$ is in $2D$, we can transform it to polar coordinates, by letting $r^2 = x^2+y^2$ and $rdrd\theta = dxdy$. We have

$$G^2 =  \int_{0}^{\infty}  \int_{0}^{2\pi} e^{-r^2} rdrd\theta.$$

Therefore on the above we can simply do the substitution $let u = r^2$ and therefore $du = 2rdr$. We now have

$$G^2 =   \int_{0}^{\infty}  \int_{0}^{2\pi} \frac{1}{2} e^{-u} dud\theta,$$

and hence

$$G^2 = - \pi e^{-u} \bigg|_{0}^{\infty} = \pi.$$

Now solving for $G$, we get that 

$$G = \sqrt{pi}.$$

We used this trick above to find the leading order behavior of the Laplace Integral, and will use it again in a minute for a similar computation! 


%
% LAPLACE'S METHOD w/ minimum on edge of integration bounds 
%
\subsubsection{Laplace's Method Take Two (minimizer on an integration bound)}

Consider (\ref{laplace_integral}), such that $g(t)$ assumes a strict minimum in $[a,b]$ at an endpoint of the interval. Without loss of generality, let's just say the minimum is found at $t=a$. We will also assume that $g'(a)=0$ (duh), $g''(a)>0$ (so the point is a minimum, duh), and $f(a)\neq 0$. Using the same math trick of adding and subtracting zero, we can rewrite (\ref{laplace_integral}) as

$$I(\lambda) = e^{-\lambda g(a)} \int_a^b f(t) e^{ -\lambda\big[ g(t) - g(a) \big] } dt.$$

We will make the same assumptions as we did above, e.g., that the main contribution to $I(\lambda)$ comes from a neighborhood of $t=a$, $f(t)$ and $g(t)$ are smooth enough so that we can replace them by their Taylor Series centered at $t=a$, and that since the main contribution is coming a neighborhood of $t=a$, we will be able to expand this neighborhood to all of $\mathbb{R}^+$ with only higher order asymptotic terms being affected. The reason for only blowing this interval up to $[a,\infty)$ is because we do not consider points $t<a$. Upon doing these things, we get

\begin{align*}
I(\lambda) &\approx e^{-\lambda g(a)} \int_{a}^{a+\epsilon} f(t) e^{ -\lambda\big[ g(t) - g(a) \big] } dt \\ \\
	& \approx  e^{-\lambda g(a)} f(a) \int_{a}^{a+\epsilon} e^{ -\lambda\big[ g(t) - g(a) \big] } dt  \\  \\
	&\approx e^{-\lambda g(a)} f(a) \int_{a}^{c+\epsilon}  e^{ -\lambda\big[ \big( g(a) +(t-a)g'(a) + \frac{1}{2!} (t-a)^2 g''(a) \big) - g(a) \big] } dt \\ \\
	& \approx  e^{-\lambda g(a)} f(a) \int_{a}^{c+\epsilon} e^{ -\lambda\big[  g'(a)(t-a) + \frac{1}{2!}  g''(a)(t-a)^2 \big] } dt  \\  \\
	&= e^{-\lambda g(a)} f(a) \int_{a}^{a+\epsilon} e^{ - \frac{\lambda}{2}  g''(a)(t-a)^2 } dt  \\  \\
	&  \approx e^{-\lambda g(a)} f(a) \int_{a}^{\infty} e^{ - \frac{\lambda}{2}  g''(a)(t-a)^2 } dt.
\end{align*}

Now doing the same substitution as before, letting $s=t-a$, we are left with the following Gaussian-type integral to integrate,

\begin{align*}
I(\lambda) &\approx e^{-\lambda g(a)} f(a) \int_{0}^{\infty} e^{ - \frac{\lambda}{2}  g''(a)s^2 } ds  \\ \\
	&= e^{-\lambda g(a) } f(a) \sqrt{ \frac{\pi}{2\lambda g''(a) } }.
\end{align*}

Thus when the minimizer is at an endpoint, we have to leading order that 

\begin{equation}
\label{endpoint_laplace} I(\lambda) \sim  e^{-\lambda g(a) } f(a) \sqrt{ \frac{\pi}{2\lambda g''(a) } } \ \ \ \mbox{ as } \lambda\rightarrow\infty.
\end{equation}


%
% THE GAMMA FUNCTION AND HIGHER ORDER ASYMPTOTICS
%
\subsubsection{The Gamma Function and Higher Order Asymptotics}

Usually in the asymptotic business, the next thing one is introduced to is the \emph{Gamma function}. The reason for this is two-fold; first it is a generalization of the factorial function, and second, it is used to derive Stirling's approximation, which is looking at the asymptotic behavior of the factorial function itself. We will do all these things, but before, we concern ourself with the asymptotic behavior of the factorial, let's use the Gamma function to look at higher order asymptotics associated with the Laplace integral. 

The Gamma function is defined as, 
\begin{equation}
\label{gamma_function} \Gamma(x) = \int_0^\infty t^{x-1} e^{-t} dt.
\end{equation}

The way to see (\ref{gamma_function}) is a continuous generalization of the factorial function, we simply have to integrate it by parts, e.g., 
\begin{align*}
\Gamma(x+1) &= \int_0^\infty t^x e^{-t} dt \\ \\
	&= - t^{x} e^{-t} \bigg|_{0}^\infty + \int_{0}^{\infty} x t^{x-1} e^{-t} dt \\ \\
	&= x \int_{0}^{\infty} t^{x-1} e^{-t} dt \\ \\
	&= x\Gamma(x).
\end{align*}

Thus if $x=n\in\mathbb{Z}^+$, we have $$\Gamma{(n+1)}=n!.$$

Now we can use this to find higher-order asymptotic terms in vein of the Laplace integral in (\ref{laplace_integral}). We will again assume  that $g(t)$ assumes a strict minimum over $[a,b]$ at a point $c\in(a,b)$, as well that $g'(c)=0, g''(c)\neq0,$ and $f''(c)\neq 0.$ Now where this calculation is different than those previously done is that we assume $f(c)=0$. This subtle change will lead us to employ the Gamma function to find the leading behavior in our asymptotic expansion. We will again make the same assumptions as before, e.g., the main contribution comes from the neighborhood of $t=c$, and all that jazz. Now for $\lambda>>1$,

\begin{align*}
I(\lambda) &=  \int_a^b f(t) e^{-\lambda g(t) } dt \\ \\
	&\approx e^{-\lambda g(c)} \int_{c-\epsilon}^{c+\epsilon} f(t) e^{-\lambda \big( g(t) - g(c) \big) } dt \\ \\
	&= e^{-\lambda g(c)} \int_{c-\epsilon}^{c+\epsilon}  \left[ f(c) + f'(c)(t-c) + \frac{1}{2!} f''(c) (t-c)^2 + \mathcal{O}( (t-c)^3 )  \right] e^{- \lambda \big( g'(c)(t-c) + \frac{1}{2!} g''(c)(t-c)^2 +  \mathcal{O}( (t-c)^3 ) \big) } dt \\ \\
	&\approx e^{-\lambda g(c)} \int_{c-\epsilon}^{c+\epsilon}  \left[  f'(c)(t-c) + \frac{1}{2!} f''(c) (t-c)^2 \right] e^{- \frac{\lambda}{2}  g''(c)(t-c)^2 } dt \\ \\
	&= \frac{f''(c)}{2} e^{-\lambda g(c) }  \int_{c-\epsilon}^{c+\epsilon} (t-c)^2 e^{-\frac{\lambda}{2} g''(c) (t-c)^2 } dt.\\
\end{align*}

We do the same transformation $s = t-c$, and then expanding the integration region to all of $\mathbb{R}$, we obtain
\begin{align*}
I(\lambda) &\approx  \frac{f''(c)}{2} e^{-\lambda g(c) }  \int_{-\infty}^{\infty} s^2 e^{-\frac{\lambda}{2} g''(c) s^2 } ds\\ \\
	&= f''(c) e^{-\lambda g(c) } \int_0^{\infty} s^2 e^{-\frac{\lambda}{2} g''(c) s^2 } ds. \\ 
\end{align*}

To simplify the above integral, we perform one more transformation. Letting $u = \frac{\lambda}{2}g''(c) s^2$ and hence $du = \lambda g''(c) s ds$, we find that
\begin{align*}
I(\lambda) &\approx f''(c) e^{-\lambda g(c) } \int_0^{\infty} \left( \frac{2u}{\lambda g''(c) } \right) e^{-u} \left( \frac{du}{2\lambda g''(c) u} \right) \\ \\
	&= f''(c) e^{-\lambda g(c) } \int_{0}^{\infty} \frac{ \sqrt{2} }{ (\lambda g''(c) )^{3/2} } u^{1/2} e^{-u} du \\ \\
	&= \frac{ \sqrt{2} f''(c) }{ (\lambda g''(c) )^{3/2} } e^{-\lambda g(c) } \Gamma\left(\frac{3}{2}\right) \\ \\
	&= \frac{ \sqrt{2} f''(c) }{ (\lambda g''(c) )^{3/2} } e^{-\lambda g(c) } \left( \frac{1}{2} \right) \Gamma\left( \frac{1}{2} \right) \\ 
 \end{align*}
 
 Now we find that $$\Gamma\left( \frac{1}{2} \right) = \sqrt{ \pi }, i.e.,$$
\begin{align*}
\Gamma\left( \frac{1}{2} \right) &= \int_0^{\infty} t^{-1/2} e^{-t} dt, \ \ \ \ \mbox{ let } u=t^{1/2} \\ \\
	&= 2 \int_0^{\infty} e^{-u^2}  du \\ \\
	&= \sqrt{\pi}. \\
\end{align*} 
 
 Hence we find that 
 $$I(\lambda) \approx \sqrt{ \frac{\pi}{2} } \frac{ f''(c) }{ ( \lambda g''(c) )^{3/2} } e^{-\lambda g(c) }.$$
 
 Therefore we have to leading order,
 
 $$I(\lambda) \sim f''(c) e^{-\lambda g(c) } \sqrt{  \frac{\pi}{ 2 \left( \lambda g''(c) \right)^3   } }, \ \ \ \ \mbox{ as } \lambda\rightarrow\infty.$$
 
 
 %
% STIRLING'S APPROXIMATION
%
\subsubsection{Striling's Approximation}
 
 We now set our sights on deciphering the leading order asymptotics of $n!$ as $n\rightarrow\infty$. To do this we will rightfully use Laplace's method, but first will motion to massaging the Gamma function into a more welcoming form for Laplace's method. Consider
 
$$\Gamma\left( x+1 \right) = \int_0^\infty t^x e^{-t} dt.$$

I hope you remember your exponential and logarithmic identities. We are about to go for a walk down memory lane. First let's warmup and use the property that $$t^x = e^{ x\ln(t) }$$,

$$\Gamma\left( x+1 \right) = \int_0^\infty e^{ x\ln(t) } e^{-t} dt = \int_0^{\infty} e^{-x \left( \frac{t}{x} - \ln(t)  \right) } dt.$$

Next we perform a substitution, $z = \frac{t}{x}$ (and $dz = \frac{1}{x} dt$), and get

$$\Gamma\left( x+1 \right) = x \int_0^\infty e^{-x \left(  z-\ln(xz) \right) } dz = x \int_0^\infty e^{-x \left(  z-\ln(x)-\ln(z) \right) } dz = xe^{x\ln(x)} \int_0^\infty e^{-x\left(  z-\ln(z)  \right) } dt.$$

We now have exponentially and logarithmically massaged the Gamma integral into the following form,

\begin{equation}
\label{stirling_gamma_int} \Gamma\left( x+1 \right) = x^{x+1} \int_0^\infty e^{-x \left( z - \ln z \right) } dz.
\end{equation}

We see that (\ref{stirling_gamma_int}) is in the form of a Laplace type integral, with $f(z)=1$ and $g(z) = z - \ln z$. With a  little further analysis, it is clear that $g(z)$ has a strict minimum on $(0,\infty)$ at $z=1$. We see that $g(z=1)=1, g'(1)=0$, and $g''(1)=1.$ Substituting all of this information into the leading order terms of the Laplace integral, we find that 

$$\int_0^\infty e^{-x\left(  z - \ln z \right) } dz \sim e^{-x g(1) } f(1) \sqrt{ \frac{2\pi}{\lambda g''(1) } }  = \sqrt{ \frac{2\pi}{x} } e^{-x}, \ \ \ \mbox{ as } x\rightarrow\infty.$$

Hence we find that the Gamma function has leading order behavior,

$$\Gamma(x+1) \sim x^{x+1} \sqrt{ \frac{2\pi}{x} } e^{-x} = \sqrt{2\pi} x^{ x+\frac{1}{2} } e^{-x}, \ \ \ \mbox{ as} x\rightarrow\infty.$$

To make this a little closer to home, consider $x=n\in\mathbb{Z}^+$, as $n\rightarrow\infty$, that is,  \emph{make n is very large integer}. We are now ready to see what the factorial function looks like as $n$ gets very, very large. 

$$n! \sim \sqrt{2\pi} n^{n+\frac{1}{2}} e^{-n}, \  \ \ \mbox{ as } n\rightarrow\infty.$$
 