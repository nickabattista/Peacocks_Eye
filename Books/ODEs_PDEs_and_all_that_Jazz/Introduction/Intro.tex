\graphicspath{{./Intro/}}

\section{Introduction}

Welcome! In this short 5 day course, we will cover a variety of topics pertaining to analytical methods in applied mathematics. Most of these topics have been chosen in part because this material that seems to have a way of falling through the cracks of most undergraduate curriculums in mathematics and because they are traditionally not covered during the first year sequence in methods of applied mathematics. 

The tentative list of topics we will cover is below: 
\begin{enumerate}
%
\item A review of undergraduate ODEs
\begin{itemize}
\item Separable Equations
\item First-Order: Integrating Factor
\item Homogeneous vs. Nonhomogeneous Equations
\item Solving for Complementary Solutions
\begin{itemize}
\item Linear constant coefficient ODEs
\end{itemize}
\item Solving for Particular Solution:
\begin{itemize}
\item Undetermined Coefficients
\item Variation of Parameters
\end{itemize}
\item Bonus Methods:
\begin{itemize}
\item Transform Methods (i.e., ``Euler" type, Ricatti type, etc)
\item Laplace Transforms
\item Non-dimensionalization
\item Introduction to Dynamical Systems, e.g., systems of ODEs
\end{itemize}
\end{itemize} 
%
\item General Sturm-Liouville Theory
\begin{itemize}
\item Orthogonal Functions
\item Fourier Series
\item Orthogonal Polynomials as a Complete Basis
\end{itemize}
%
\item Greens Functions
%
\item Linear PDEs
\begin{itemize}
\item Classification of PDEs
\item Derivations of classical linear PDEs
\item Method of Characteristics
\item Separation of Variables
\item Fourier Transforms / Laplace Transforms
\item Similarity Solutions
\item Method of Images
\end{itemize}
%
\item Physics!
\begin{itemize}
\item Newton's Laws of Motion
\item Conservation Laws
\item Lagrangian Formulation of Mechanics
\item Hamiltonian Formulation of Mechanics
\end{itemize}
%
\item Big Picture Numerical Linear Algebra
\begin{itemize}
\item Linear systems, the bread and butter.
\item Nonlinear systems, root-finding, and iterative methods - oh my!
\item Eigen-fun: a story of eigenvalues, eigenvectors, and love.
\end{itemize}
%
\end{enumerate}

Please note that these notes may not be complete at this time. Please check back periodically as more sections will be added...\emph{when my code is compiling.} 