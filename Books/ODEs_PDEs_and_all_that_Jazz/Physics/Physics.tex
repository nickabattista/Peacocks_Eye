\graphicspath{{./Physics/}}

\section{Physics!}

\emph{You're taking physics? I love physics!}, said only $5\%$ of the population, of which $90\%$ were being ironic. It's no surprise that physics gets a bad rap somewhere between birth and college graduation; however, it is truly a beautiful science and arguably without it, modern mathematics would not be where it is today with it. For much of mathematics conception, math and physics (and other sciences for that matter) were like peas and carrots. It has only been within the last century or so, where people starting differentiating between calling themselves scientists to mathematicians, physicists, chemists, or biologists. I digress. Physics has a place, and from it come a lot of interesting and useful mathematics!\\

In this section we will give a crash course in basic physics ideas, e.g., conservation laws and basic mechanics, which will then lead us down the rabbit hole of \emph{calculus of variations} to learn Lagrangian and Hamiltonian formulations of mechanics. I apologize in advance for attempting to sum up years worth of courses in a minimal amount of words. There will be (many) things left out, so please do some light (gravity pun?) reading for further insight and intuition. Buckle up, we're about to see math in action!\\


%%%%%%%%%%%%%%%%%%%%%%%%%%%%%%%%%%
%
% SECTION: BASIC PHYSICS PRINCIPLES
%
%%%%%%%%%%%%%%%%%%%%%%%%%%%%%%%%%%

\subsection{Basic Physics Principles}

There once lived a man named Isaac Newton. He may have invented calculus (or co-discovered at the same time as his rival Gottfried Leibniz). What led him to calculus was his fascination with phenomena in the natural world and wanting to quantify the beauty he saw through formulae and equations. Upon doing so, he eventual wrote \emph{Philosophiae Naturalis Principia Mathematica}, which was to be a large collection of how the universe operated. Whilst his law of gravitation did not hold up past two centuries, although it is a very good approximation in the non-relativistic regime, his laws of motion remain infamous, especially with students being introduced to physics for the very first time. \\

Newton's three laws of motion are as follows
\begin{enumerate}
\item  An object stays at rest (or continues to move at constant velocity) unless an external force acts upon it.
\item The sum of all external forces acting on an object is equal to the product of the object's mass and acceleration.
\item When a body exerts a force on a second body, the second body simultaneously exerts an equal in magnitude, but opposite in direction, force on the first body.
\end{enumerate}

We will now give a more mathematical flavor of the three laws of motion.\\

\begin{itemize}

%
% LAW 1
%
\item[] {\bf{Law 1}}: \\ \\

The first law basically states if the sum of all forces on an object are zero, then the object is either at rest or moving at a constant velocity, e.g., it is not accelerating. More specifically it says that the \emph{vector} sum of all forces is zero. This law is better known as the \emph{law of intertia}. \\ 

Mathematically it says the following two statements are equivalent,
$$\sum_{j=0}^N {\bf{F}}_j = 0  \Leftrightarrow \frac{d {\bf{v}}}{dt} = 0,$$

where ${\bf{F}}_j$ is a force acting on a body and ${\bf{v}}$ is the velocity of such body. These statements say that if the body is in motion, it will remain in motion, moving in a straight line, until other forces act upon it. Furthermore, this is evident in what is referred to as the \emph{Newtonian reference frame}, or \emph{intertial} reference frame, where the motion of a body according to another observer remains moving at constant speed in a straight line. \\

%
% LAW 2
%
\item[] {\bf{Law 2}}: \\ \\

Newton's second law of motion is really a statement about the \emph{conservation of momentum}. In a nutshell it states that the time rate of change of linear momentum is equal to the sum of  forces acting on the body in an inertial reference frame. This can be described mathematically as

$$\frac{d{\bf{p}}}{dt} = \frac{ d(m{\bf{v}})}{dt} = \sum_{j=0}^N {\bf{F}}_j,$$

where ${\bf{p}}$ is the linear momentum, and is equal to mass times velocity, i.e., ${\bf{p}} = m{\bf{v}}.$ We note that this is consistent with the first law, in that if the sum of vector forces acting on a body is zero, then the momentum of the object remains unchanged. A consequence of any net force is a change in momentum of a body. \\

It is common to assume that the mass of a body does not change and is constant, leading to the infamous $$F_{net} = \sum_{j=0}^N {\bf{F}}_j = m \frac{d{\bf{v}}}{dt} = m \frac{d^2 {\bf{x}}}{dt^2} = m{\bf{a}};$$

however if mass does change, we need to consider $$F_{net}  + {\bf{u}} \frac{dm}{dt}  = \frac{d{\bf{v}}}{dt},$$ where ${\bf{u}}$ is the velocity of the escaping (or incoming) mass of the body. The extra term, ${\bf{u}} \frac{dm}{dt}$ is sometimes referred to as the advection of momentum. We note that if the body's mass is changing we do not simply treat the original statement as a product rule, i.e., $$\mbox{WRONG} \Rightarrow \ \ {\bf{F}}_{net} = \frac{d}{dt}(m{\bf{v}}) = m\frac{d{\bf{v}}}{dt} + {\bf{v}}\frac{dm}{dt} \ \ \ \Leftarrow \mbox{WRONG}.$$

The reason for this is that it does not respect Galilean invariance, or that the laws of motion must remain the same regardless of what inertial system you are in, e.g., our statement above would violate that in one frame of reference ${\bf{F}}=0$ while in another ${\bf{F}}\neq 0$. For example, say in one reference frame you are moving at constant velocity $U$, while we consider another non-moving (stationary) reference frame. We can relate the physics between the two using Galilean transformations, i.e., 
\begin{align*}
x' &= x - Ut \ \ \ \ \mbox{(position)} \\
v' &= \frac{dx'}{dt} = \frac{dx}{dt} - U \\
a' &= \frac{d^2 x'}{dt^2} = \frac{d^2 x}{dt^2} = a, \\
\end{align*} 

where all coordinates with primes indicate the moving inertial reference frame. Hence in both frames of reference we have the same acceleration, and hence conservation of momentum is preserved. \\

Moreover, if we wish to consider Newton's second law of motion for relativistic systems, we need to change our definition of linear momentum, since at speeds ``near-ish" to the speed of light, linear momentum is not simply the product of mass and velocity. \\

Newton's second law of motion is very powerful, and is the law that most people refer to the most when considering physical phenomena. For example, the main equations of fluid dynamics, called the \emph{Navier-Stokes} equations, although nonlinear PDEs, are just a statement about the conservation of momentum for a fluid. At the heart of basic mechanics, Newton's second law is main driver to solutions. 

%
% LAW 3
%
\item[] {\bf{Law 3}}: \\ \\

The crux of Newton's third law is that no force is only unidirectional. If body A exerts a force on body B, body B exerts the same force back on body A. All forces must be interactions between bodies, and no one body can magically exert a force without feeling the equal and opposite force back upon itself. \\

Think about punching a wall. You may think you have exerted all the force onto the wall, but the reason your hand hurts is because the wall has ``hit your hand" back with as much force as you placed against it. To every action, there is an equal and opposite reaction. 

\end{itemize}

We will now briefly discuss some conversation laws, or integrals of the system, in physics.



%%%%%%%%%%%%%%%%%%%%%%%%%%%%%%%%%
%
% Conservation Laws
% 
%%%%%%%%%%%%%%%%%%%%%%%%%%%%%%%%%

\subsubsection{Conservation Laws}

If a system is closed, e.g., \emph{isolated}, and the system does not interact with its background environment in any way, certain quantities arise, which are preserved throughout the entire evolution of the system considered. These quantities are known as integrals of the system, constraints of the motion, or conserved quantities. In physics, we call them conservation laws, and there are three:

\begin{enumerate}
\item Conservation of Momentum 
\item Conservation of Energy
\item Conservation of Angular Momentum
\item Conservation of Charge
\end{enumerate}

We will now briefly describe each of these conservation laws. Well not, conservation of charge, but that should be somewhat self explanatory for the level of physics we are shooting for. Note, in mathematics the term conservation law usually refers to transport type hyperbolic equations of the form, here in their scalar-ized glory, $$\partial_t u + \partial_x f(u) = 0.$$

$u$ is called the conserved quantity, while $f$ is called the flux.  

\begin{itemize}

%
%  Conservation of Momentum
% 
\item[] {\bf{Conservation of Momentum}} \\ \\

We have already seen a bit about the conservation of momentum coming from Newton's second law of motion. Rather than think about forces causing changes in momentum of a body, let's consider the momentum in a system as a whole. If one part of the system gains momentum to go one way in the system, another part of the system must have gained equal momentum, but to go in the opposite direction. \\

If we consider our system to have two particles in it, we assume that at time, $t_1$, and later time $t_2$, the total momentum of the system must be the same, e.g.,

$$\Big[ m_1 {\bf{v}}_1 + m_2 {\bf{v}}_2\Big]\ \Bigg|_{t=t_1} = \Big[ m_1 {\bf{v}}_1 + m_2 {\bf{v}}_2\Big] \ \Bigg|_{t=t_2}.$$ 

However, the above assumes that if there have been any collisions between the balls, that only \emph{elastic} collisions have occurred. That is to say, no energy has been lost, well converted, to heat, sound, etc. Spoiler alert - energy cannot be created nor destroyed, merely converted to other forms of energy. Those type of collisions where energy has been converted to heat (sound, etc.) are called \emph{inelastic} collisions. We will discuss this further in the conservation of energy section. 

Again, the main statement about the conservation of momentum is best summed up (heyyo!) through Newton's second law,
$$m \frac{d{\bf{v}}}{dt} = \sum_{j} {\bf{F}}_j.$$

Let's do some mechanics problems to illustrate the power of this law! In doing so, we will also try to motivate intuitive ideas on how to approach such mechanics problems.








\end{itemize}